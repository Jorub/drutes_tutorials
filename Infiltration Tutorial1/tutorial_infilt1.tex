%%%%%%%%%%%%%%%%%%%%%%%%%%%%%%%%%%%%%%%%%
% Arsclassica Article
% LaTeX Template
% Version 1.1 (05/01/2015)
%
% This template has been downloaded from:
% http://www.LaTeXTemplates.com
%
% Original author: Adriano Palombo
%
%
%%%%%%%%%%%%%%%%%%%%%%%%%%%%%%%%%%%%%%%%%

%----------------------------------------------------------------------------------------
%	PACKAGES AND OTHER DOCUMENT CONFIGURATIONS
%----------------------------------------------------------------------------------------

\documentclass[
10pt, % Main document font size
a4paper, % Paper type, use 'letterpaper' for US Letter paper
oneside, % One page layout (no page indentation)
%twoside, % Two page layout (page indentation for binding and different headers)
headinclude,footinclude, % Extra spacing for the header and footer
BCOR5mm, % Binding correction
]{scrartcl}

\input{structure.tex} % Include the structure.tex file which specified the document structure and layout

\hyphenation{Fortran hy-phen-ation} % Specify custom hyphenation points in words with dashes where you would like hyphenation to occur, or alternatively, don't put any dashes in a word to stop hyphenation altogether

%----------------------------------------------------------------------------------------
%	TITLE AND AUTHOR(S)
%----------------------------------------------------------------------------------------

\title{\normalfont\spacedallcaps{$DRUtES$ }} % The article title

\author{\spacedlowsmallcaps{Tutorial: standard Richards equation module: infiltration - part 1}} % The article author(s) - author affiliations need to be specified in the AUTHOR AFFILIATIONS block

%\date{} % An optional date to appear under the author(s)

%----------------------------------------------------------------------------------------

\begin{document}

%----------------------------------------------------------------------------------------
%	HEADERS
%----------------------------------------------------------------------------------------

\renewcommand{\sectionmark}[1]{\markright{\spacedlowsmallcaps{#1}}} % The header for all pages (oneside) or for even pages (twoside)
%\renewcommand{\subsectionmark}[1]{\markright{\thesubsection~#1}} % Uncomment when using the twoside option - this modifies the header on odd pages
\lehead{\mbox{\llap{\small\thepage\kern1em\color{halfgray} \vline}\color{halfgray}\hspace{0.5em}\rightmark\hfil}} % The header style

\pagestyle{scrheadings} % Enable the headers specified in this block

%----------------------------------------------------------------------------------------
%	TABLE OF CONTENTS & LISTS OF FIGURES AND TABLES
%----------------------------------------------------------------------------------------

\maketitle % Print the title/author/date block

\setcounter{tocdepth}{2} % Set the depth of the table of contents to show sections and subsections only

\tableofcontents % Print the table of contents

\listoffigures % Print the list of figures

%\listoftables % Print the list of tables

%----------------------------------------------------------------------------------------
%	AUTHOR AFFILIATIONS
%----------------------------------------------------------------------------------------

{\let\thefootnote\relax\footnotetext{* \textit{Department of Water Resources and Environmental Modeling, Faculty of Environmental Sciences, Czech University of Life Sciences}}}

%{\let\thefootnote\relax\footnotetext{\textsuperscript{1} \textit{Department of Chemistry, University of Examples, London, United Kingdom}}}

%----------------------------------------------------------------------------------------

\newpage % Start the article content on the second page, remove this if you have a longer abstract that goes onto the second page

%----------------------------------------------------------------------------------------
%	INTRODUCTION
%----------------------------------------------------------------------------------------
\section{Goal and Complexity}
\subsection*{Complexity: Beginner}

\subsection*{Prerequisites: None}

The goal of this tutorial is to get familiar with the $DRUtES$ standard Richards equation module and $DRUtES$ configuration in 1D by simulating infiltration into different soil \medskip

\begin{figure}[!h]
\centering
\includegraphics[width=8cm]{heat_cond_simple.png}
\end{figure}
The process of infiltration is fundamental and yet very important in soil science. Infiltration into the soil determines water, heat and contaminant transport. Infiltration experiments can be used to determine some parameters describing soil hydraulic properties. 

In this tutorial three configuration files will be modified step by step. All configuration files are located in the folder \emph{drutes.conf} and respective subfolders. \begin{enumerate}
\item For selection of the module, dimension and time information we require \emph{global.conf}.  \emph{global.conf} is located in \emph{drutes.conf / global.conf}. 
\item To define the mesh or spatial discretization in 1D,  we require \emph{drumesh1D.conf}. \emph{drumesh1D.conf} is located in \emph{drutes.conf / mesh / drumesh1D.conf}. 
\item To define the infiltration, we require \emph{matrix.conf}. \emph{matrix.conf} is located in \emph{drutes.conf /water.conf/ matrix.conf}. 
\end{enumerate}
$DRUtES$ works with configuration input file with the file extension .conf. Blank lines and lines starting with \# are ignored. The input mentioned in this tutorial therefore needs to be placed one line below the mentioned keyword, unless stated otherwise. 

\section{Software}

\begin{enumerate}
\item Install $DRUtES$. You can get $DRUtES$ from the github repository \href{https://github.com/michalkuraz/drutes-dev/} {drutes-dev} or download it from the \href{http://drutes.org/public/?core=account}{drutes.org} website. 
\item Follow website instructions on \href{http://drutes.org/public/?core=account}{drutes.org} for the installation.
\item Working R installation (optional, to generate plots you can execute freely distributed R script) 
\end{enumerate}

\newpage
\section{Scenarios}

We are using the well-known van Genuchten-Mualem parameterization to describe the soil hydraulic properties of our soils. 

\begin{table}[!h]\caption{\label{tab_heat}Material properties needed for scenarios.}
\adjustbox{max height=\dimexpr\textheight-5cm\relax,
           max width=\textwidth}{

\small\begin{tabular}{l c c c c c c}
\hline
& $\alpha$ & n & m &$ \theta_s$ &$ \theta_r$ & $K_s$ \\
& &   & && \\
Material & &   & &&\\
 \hline
Sand &   & &&\\
Silt & &   & && \\
Clay &  &   & && \\
\hline
\end{tabular}
}
\end{table}


\section*{Scenario 1}

Infiltration into sandy soil. 

$global.conf$: Choose correct model, dimension, time discretization and observation times.
\begin{enumerate}
\item Open \textbf{\emph{global.conf}} in a text editor of your choice. 
\item Model type: Your first input is the module. Input is \textbf{RE\_stdH}.
\item Initial mesh configuration \begin{enumerate}
\item The dimension of our problem is 1. Input: 1.
\item We use the internal mesh generator. Input: 1. 
\end{enumerate}
\item Error criterion (not needed here, leave at default value) \begin{enumerate} 
\item Maximum number of iteration of the Picard method: 20 
\item h tolerance: 1e-2.
\end{enumerate}
\item Time information 
\begin{enumerate} 
%\item integration method is 3 point formula. Input: 30. 
\item Time units are in hours: input h
\item Initial time: 1e-3.
\item End time: 24.
\item Minimum time step: 1e-6.
\item Maximum time step: 0.1.
\end{enumerate}
\item Observation time settings \begin{enumerate}
\item Observation time method: 2
\item Set file format of observation: pure. Output in 1D is always in raw data. Different options will not impact output in 1D.
\item Make sequence of observation time: n
\item Number of observation times: 11
\item Observation time values: 2, 4, 6, 8, 10, 12 ,14, 16, 18, 20, 22. Use a new line for each input. \textit{DRUtES} automatically generates output for the initial time and final time. DRUtES will generate 13 output files, e.g. \textit{MISSINS.dat}, where x is the number of the file and not the output time. The initial time is assigned an x value of 0. 
\end{enumerate}
\item Observation point settings \begin{enumerate}
\item Observation point coordinates: 0.0, 1. Use a new line for each input. \textit{DRUtES} will generate 2 output files, e.g. \textit{MISSING}, where x is the ID of the observation point. 
\end{enumerate}
\item Ignore other settings for now. 
\item Save $global.conf$
\end{enumerate}


$drumesh1D.conf$: Mesh definition, i.e. number of materials and spatial discretization
\begin{enumerate}
\item Open \textbf{\emph{drumesh1D.conf}} in a text editor of your choice. 
\item Geometry information: 0.2 m - domain length
\item Amount of intervals: 1
\item
\adjustbox{max height=\dimexpr\textheight-5cm\relax,
           max width=\textwidth}{
\small\begin{tabular}{|c | c | c|}
\hline
density & bottom & top \\
 \hline
0.05 & 0 & 1\\
\hline
\end{tabular}
}
\item number of materials: 1
\item \adjustbox{max height=\dimexpr\textheight-5cm\relax,
           max width=\textwidth}{
\small\begin{tabular}{|c | c | c|}
\hline
id & bottom & top \\
 \hline
1& 0 & 1 \\
\hline
\end{tabular}
}
\end{enumerate}

\emph{heat.conf}: Heat module after Sophocleous (1979). 


\begin{enumerate}
\item Open \emph{matrix.conf} in a text editor of your choice. 
\item Save matrix.conf.

\end{enumerate}

\section*{Run scenario 1}
Run the simulation in the terminal console.
\begin{enumerate}
\item Make sure you are in the right directory. 
\item To execute $DRUtES$: \\
\$ bin/drutes
\item After the simulation finishes, to generate png plots execute provided R script: \\
\$ Rscript drutes.conf/water.conf/drutesplots.R sand
\item The output of the simulation can be found in the folder out
\end{enumerate}

\section*{Tasks for scenario 1}

\begin{enumerate}
\item Q1
\item Q2
\item Q3
\end{enumerate}


\section{Outcome}
\begin{enumerate}
\item You got familiar with the $DRUtES$ standard Richards Equation modules in 1D.
\item You understand basic parameterization of a typical sand, silt and clay with the van Genuchten-Mualem model.
\item You simulated infiltration in different soils.
\item You understand the term \emph{Neumann boundary condition} and \emph{initial condition}.
\end{enumerate}

%----------------------------------------------------------------------------------------
%	BIBLIOGRAPHY
%----------------------------------------------------------------------------------------

%\renewcommand{\refname}{\spacedlowsmallcaps{References}} % For modifying the bibliography heading

%\bibliographystyle{unsrt}

%\bibliography{sample.bib} % The file containing the bibliography

%----------------------------------------------------------------------------------------

\end{document}