%%%%%%%%%%%%%%%%%%%%%%%%%%%%%%%%%%%%%%%%%
% Arsclassica Article
% LaTeX Template
% Version 1.1 (05/01/2015)
%
% This template has been downloaded from:
% http://www.LaTeXTemplates.com
%
% Original author: Adriano Palombo
%
%
%%%%%%%%%%%%%%%%%%%%%%%%%%%%%%%%%%%%%%%%%

%----------------------------------------------------------------------------------------
%	PACKAGES AND OTHER DOCUMENT CONFIGURATIONS
%----------------------------------------------------------------------------------------

\documentclass[
10pt, % Main document font size
a4paper, % Paper type, use 'letterpaper' for US Letter paper
oneside, % One page layout (no page indentation)
%twoside, % Two page layout (page indentation for binding and different headers)
headinclude,footinclude, % Extra spacing for the header and footer
BCOR5mm, % Binding correction
]{scrartcl}

\input{structure.tex} % Include the structure.tex file which specified the document structure and layout

\hyphenation{Fortran hy-phen-ation} % Specify custom hyphenation points in words with dashes where you would like hyphenation to occur, or alternatively, don't put any dashes in a word to stop hyphenation altogether

%----------------------------------------------------------------------------------------
%	TITLE AND AUTHOR(S)
%----------------------------------------------------------------------------------------

\title{\normalfont\spacedallcaps{$DRUtES$ }} % The article title

\author{\spacedlowsmallcaps{Tutorial: coupled heat and water - part 1}} % The article author(s) - author affiliations need to be specified in the AUTHOR AFFILIATIONS block

%\date{} % An optional date to appear under the author(s)

%----------------------------------------------------------------------------------------

\begin{document}

%----------------------------------------------------------------------------------------
%	HEADERS
%----------------------------------------------------------------------------------------

\renewcommand{\sectionmark}[1]{\markright{\spacedlowsmallcaps{#1}}} % The header for all pages (oneside) or for even pages (twoside)
%\renewcommand{\subsectionmark}[1]{\markright{\thesubsection~#1}} % Uncomment when using the twoside option - this modifies the header on odd pages
\lehead{\mbox{\llap{\small\thepage\kern1em\color{halfgray} \vline}\color{halfgray}\hspace{0.5em}\rightmark\hfil}} % The header style

\pagestyle{scrheadings} % Enable the headers specified in this block

%----------------------------------------------------------------------------------------
%	TABLE OF CONTENTS & LISTS OF FIGURES AND TABLES
%----------------------------------------------------------------------------------------

\maketitle % Print the title/author/date block

\setcounter{tocdepth}{2} % Set the depth of the table of contents to show sections and subsections only

\tableofcontents % Print the table of contents

\listoffigures % Print the list of figures

%\listoftables % Print the list of tables

%----------------------------------------------------------------------------------------
%	AUTHOR AFFILIATIONS
%----------------------------------------------------------------------------------------

{\let\thefootnote\relax\footnotetext{* \textit{Department of Water Resources and Environmental Modeling, Faculty of Environmental Sciences, Czech University of Life Sciences}}}

%{\let\thefootnote\relax\footnotetext{\textsuperscript{1} \textit{Department of Chemistry, University of Examples, London, United Kingdom}}}

%----------------------------------------------------------------------------------------

\newpage % Start the article content on the second page, remove this if you have a longer abstract that goes onto the second page

%----------------------------------------------------------------------------------------
%	INTRODUCTION
%----------------------------------------------------------------------------------------
\section{Goal and Complexity}
\subsection*{Complexity: Medium}

\subsection*{Prerequisites: None}

The goal of this tutorial is to get familiar with the idea of coupled models in 1D. For this we couple the $DRUtES$ standard Richards equation module and the heat module. We apply solar radiation data and rain and evaporation data as input.

\begin{figure}[!h]
\centering
\includegraphics[width=8cm]{coupled.png}
\caption{Simplified scheme of coupled model.}
\end{figure}

In the real world many coupled processes occur. Heat conduction is dependent on the water content and water flow is dependent on heat properties. 

In this tutorial three configuration files will be modified step by step. All configuration files are located in the folder \emph{drutes.conf} and respective subfolders. \begin{enumerate}
\item For selection of the module, dimension and time information we require \emph{global.conf}.  \emph{global.conf} is located in \emph{drutes.conf / global.conf}. 
\item To define the mesh or spatial discretization in 1D,  we require \emph{drumesh1D.conf}. \emph{drumesh1D.conf} is located in \emph{drutes.conf / mesh / drumesh1D.conf}. 
\item To define the radiation and cooling, we require \emph{heat.conf}. \emph{heat.conf} is located in \emph{drutes.conf /heat/ heat.conf}. 
\item To define the precipitation and evaporation, we require \emph{matrix.conf}. \emph{matrix.conf} is located in \emph{drutes.conf /water.conf/ matrix.conf}. 
\end{enumerate}
$DRUtES$ works with configuration input file with the file extension .conf. Blank lines and lines starting with \# are ignored. The input mentioned in this tutorial therefore needs to be placed one line below the mentioned keyword, unless stated otherwise. 

\section{Software}

\begin{enumerate}
\item Install $DRUtES$. You can get $DRUtES$ from the github repository \href{https://github.com/michalkuraz/drutes-dev/} {drutes-dev} or download it from the \href{http://drutes.org/public/?core=account}{drutes.org} website. 
\item Follow website instructions on \href{http://drutes.org/public/?core=account}{drutes.org} for the installation.
\item Working R installation (optional, to generate plots you can execute freely distributed R script) 
\end{enumerate}

\newpage
\section{Scenarios}

We are using the well-known van Genuchten-Mualem parameterization to describe the soil hydraulic properties of our soils. 

\begin{table}[!h]
\centering
\caption{\label{tab_heat}Material properties needed for scenarios.}
\adjustbox{max height=\dimexpr\textheight-2cm\relax,
           max width=\textwidth}{

\small\begin{tabular}{l l c c }
\hline
Parameter & Description & Soil \\
\hline

$\alpha$ [cm$^{-1}$]& inverse of the air entry value &0.05 \\
 $n$ [-]& shape parameter &2 \\
 $m$ [-]& shape parameter &0.5  \\
  $ \theta_s$ [-]&saturated vol. water content&0.45 \\
  $ \theta_r$ [-]& residual vol. water content&0.05  \\
  $Ss$ [cm$^{-1}$] & specific storage & 0 \\
  $K_s$ [cm d$^{-1}$] & saturated hydraulic conductivity & 100 \\ 
   $c_l$ [Wd cm$^{-3}$ K$^{-1}$ ] & specific heat capacity of soil & 2.545e-5 \\
  $c_w$ [Wd cm$^{-3}$ K$^{-1}$ ] & specific heat capacity of water & 4.843e-5	 \\ 
  $\lambda$ [W cm$^{-1}$ K$^{-1}$] & thermal conductivity & 0.02 \\ 
\hline
\end{tabular}
}
\end{table}


\begin{figure}[!h]
\centering
\includegraphics[width=10cm]{domain_coupled.png}
\caption{1D domain set-up of coupled scenario with top and bottom boundary conditions. There are now two boundary condtions at the top and two boundary conditions at the bottom: heat and water flow. The top boundary is defined by the interactions with the atmosphere and the bottom boundary is defined by the constant groundwater table.}
\end{figure}


\begin{figure}[!h]
\centering
\includegraphics[width=10cm]{radiation_data.png}
\caption{Heat flow data used for the top boundary}
\end{figure}

\begin{figure}[!h]
\centering
\includegraphics[width=10cm]{water_data.png}
\caption{Water flow data used for the top boundary}
\end{figure}
\newpage

\section*{Scenario 1}

Coupled model

$global.conf$: Choose correct model, dimension, time discretization and observation times.
\begin{enumerate}
\item Open \textbf{\emph{global.conf}} in a text editor of your choice. 
\item Model type: Your first input is the module. Input is \textbf{heat}.
\item Initial mesh configuration \begin{enumerate}
\item The dimension of our problem is 1. Input: 1.
\item We use the internal mesh generator. Input: 1. 
\end{enumerate}
\item Error criterion \begin{enumerate} 
\item Maximum number of iteration of the Picard method: 20 
\item h tolerance: 1.
\end{enumerate}
\item Time information 
\begin{enumerate} 
%\item integration method is 3 point formula. Input: 30. 
\item Time units are in hours: input d
\item Initial time: 1e-6.
\item End time: 14.
\item Minimum time step: 1e-6.
\item Maximum time step: 0.001.
\end{enumerate}
\item Observation time settings \begin{enumerate}
\item Observation time method: 2
\item Set file format of observation: pure. Output in 1D is always in raw data. Different options will not impact output in 1D.
\item Make sequence of observation time: n
\item Number of observation times: 0
\item Observation time values: \#
\end{enumerate}
\item Observation point settings \begin{enumerate}
\item Number of observation points: 6 
\item Observation point coordinates: 200, 195,180,160,140, 120. Use a new line for each input. \textit{DRUtES} will generate 6 output files, e.g. \textit{obspt\_RE\_matrix-1.out}, where x is the ID of the observation point. 
\end{enumerate}
\item Ignore other settings for now. 
\item Save $global.conf$
\end{enumerate}


$drumesh1D.conf$: Mesh definition, i.e. number of materials and spatial discretization
\begin{enumerate}
\item Open \textbf{\emph{drumesh1D.conf}} in a text editor of your choice. 
\item Geometry information: 200 cm - domain length
\item Amount of intervals: 1
\item
\adjustbox{max height=\dimexpr\textheight-5cm\relax,
           max width=\textwidth}{
\small\begin{tabular}{|c | c | c|}
\hline
density & bottom & top \\
 \hline
4 & 0 & 200\\
\hline
\end{tabular}
}
\item number of materials: 1
\item \adjustbox{max height=\dimexpr\textheight-5cm\relax,
           max width=\textwidth}{
\small\begin{tabular}{|c | c | c|}
\hline
id & bottom & top \\
 \hline
1 & 0 & 200 \\
\hline
\end{tabular}
}
\end{enumerate}

\emph{matrix.conf}: Configuration file for water flow 


\begin{enumerate}
\item Open \emph{matrix.conf} in a text editor of your choice. 
\item How-to use constitutive relations? [integer]: 1
\item Length of interval for precalculating the constitutive functions: 200
\item Discretization step for constituitive function precalculation: 0.1
\item number of soil layers [integer]: 1
\item \adjustbox{max height=\dimexpr\textheight-5cm\relax,
	max width=\textwidth}{
	\small\begin{tabular}{|c | c | c|c | c | c|}
		\hline
		alpha & n & m & theta\_r & theta\_s & specific storage\\
		\hline
		0.05 & 2  & 0.5 & 0.05 & 0.45  &0  \\
		\hline
	\end{tabular}
}
\item The angle of the anisotropy determines the angle of the reference coordinate system. 0 means vertical flow. Anisotropy description. Anisotpropy description and hydraulic conductivity\\ \adjustbox{max height=\dimexpr\textheight-5cm\relax,
	max width=\textwidth}{
	\small\begin{tabular}{|c | c |}
		\hline
		angle [degrees] & K\_11  \\
		\hline
		0 & 100  \\
		\hline
	\end{tabular}
}
\item sink(-) /source (+) term per layer: 0
\item Initial condition is a constant pressure head of -200 cm across the soil. \adjustbox{max height=\dimexpr\textheight-5cm\relax,
	max width=\textwidth}{
	\small\begin{tabular}{|c | c | c|c |}
		\hline
		 init. cond [real] & type of init. cond &RCZA method [y/n] &  RCZA method val.  \\
		\hline
		   0.0           &            H\_tot                      & n		   &          0 \\
		\hline
	\end{tabular}
}
\item number of boundaries: 2
\item \adjustbox{max height=\dimexpr\textheight-5cm\relax,
	max width=\textwidth}{
	\small\begin{tabular}{|c | c | c|c |}
		\hline
		boundary ID& boundary type   & use rain.dat [y/n]   & value  \\
		\hline
	101    &                   1        &           n        &        0.0 \\
	102             &          2         &          y         &       0.0        \\
	
		\hline
	\end{tabular}
}
\item Save matrix.conf.
\end{enumerate}

\emph{heat.conf}: Heat module after Sophocleous (1979). 


\begin{enumerate}
\item Open \emph{heat.conf} in a text editor of your choice. 

\item Couple with Richards equation: y
\item Number of materials or layers: 1 
\item Specific heat capacity of the wall material:  2.545e-5 Wd cm$^{-3}$ K$^{-1}$ 
\item Specific heat capacity of liquid: 4.843e-5  Wd cm$^{-3}$ K$^{-1}$ 
\item Anisotropy: There is no anisotropy. The value is 0.
\item Heat conductivity of the soil material: 0.02 W cm$^{-1}$ K$^{-1}$. 
\item There is NO heat convection of water: 0.
\item The initial temperature is 0$^{\circ}$C across the entire domain: 0.
\item There is no heat source: 0. 
\item We have 2 boundaries at top and bottom of the soil column. We assume a constant temperature of 15 $^{\circ}$C at the bottom where the groundwater table is. We assume the the top to be influenced by radiation and cooling, which are both flux or Neumann conditions. \\
\adjustbox{max height=\dimexpr\textheight-5cm\relax,
           max width=\textwidth}{
\small\begin{tabular}{|c | c | c| c|}
\hline
boundary id & boundary type & use bc.dat & value \\
 \hline
101& 1 & n & 15.0 \\
102& 2 & y & 0.0 \\
\hline
\end{tabular}
}

\item Save heat.conf.
\end{enumerate}

\section*{Run scenario 1}
Run the simulation in the terminal console.
\begin{enumerate}
\item Make sure you are in the right directory. 
\item To execute $DRUtES$: \\
\$ bin/drutes
\item After the simulation finishes, to generate png plots execute provided R script: \\
\$ Rscript drutes.conf/water.conf/waterplots.R -name coupled \\
\$ Rscript drutes.conf/heat/heatplots.R coupled
\item The output of the simulation can be found in the folder out
\end{enumerate}


\section*{Tasks}

\begin{enumerate}
\item Describe the temperature and water content distribution.
\end{enumerate}



\section*{Results}

The water content at the top follows the flux input. The lower the observation point, the less fluctuations can be observed.
The temperature follows the heat flux input. The temperature fluctuations become smaller with depth, but also show a lag time.

\begin{figure}[!h]
\centering
\includegraphics[width=10cm]{obs_water_point_coupled_standard.png}
\caption{Water content at the observation points}
\end{figure}

\begin{figure}[!h]
\centering
\includegraphics[width=10cm]{obs_temp_point_temp_coupled_standard.png}
\caption{Temperature at the observation points.}
\end{figure}

\newpage
\newpage

\section{Outcome}
\begin{enumerate}
\item You got familiar with the idea of coupled models. 
\item You simulated coupled water flow and heat flow.
\item You understand the influence of distance to input flux and how the top is the most varying layer.
\end{enumerate}



%----------------------------------------------------------------------------------------
%	BIBLIOGRAPHY
%----------------------------------------------------------------------------------------

%\renewcommand{\refname}{\spacedlowsmallcaps{References}} % For modifying the bibliography heading

%\bibliographystyle{unsrt}

%\bibliography{sample.bib} % The file containing the bibliography

%----------------------------------------------------------------------------------------

\end{document}